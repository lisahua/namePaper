% This is "sig-alternate.tex" V2.0 May 2012
% This file should be compiled with V2.5 of "sig-alternate.cls" May 2012
%
% This example file demonstrates the use of the 'sig-alternate.cls'
% V2.5 LaTeX2e document class file. It is for those submitting
% articles to ACM Conference Proceedings WHO DO NOT WISH TO
% STRICTLY ADHERE TO THE SIGS (PUBS-BOARD-ENDORSED) STYLE.
% The 'sig-alternate.cls' file will produce a similar-looking,
% albeit, 'tighter' paper resulting in, invariably, fewer pages.
%
% ----------------------------------------------------------------------------------------------------------------
% This .tex file (and associated .cls V2.5) produces:
%       1) The Permission Statement
%       2) The Conference (location) Info information
%       3) The Copyright Line with ACM data
%       4) NO page numbers
%
% as against the acm_proc_article-sp.cls file which
% DOES NOT produce 1) thru' 3) above.
%
% Using 'sig-alternate.cls' you have control, however, from within
% the source .tex file, over both the CopyrightYear
% (defaulted to 200X) and the ACM Copyright Data
% (defaulted to X-XXXXX-XX-X/XX/XX).
% e.g.
% \CopyrightYear{2007} will cause 2007 to appear in the copyright line.
% \crdata{0-12345-67-8/90/12} will cause 0-12345-67-8/90/12 to appear in the copyright line.
%
% ---------------------------------------------------------------------------------------------------------------
% This .tex source is an example which *does* use
% the .bib file (from which the .bbl file % is produced).
% REMEMBER HOWEVER: After having produced the .bbl file,
% and prior to final submission, you *NEED* to 'insert'
% your .bbl file into your source .tex file so as to provide
% ONE 'self-contained' source file.
%
% ================= IF YOU HAVE QUESTIONS =======================
% Questions regarding the SIGS styles, SIGS policies and
% procedures, Conferences etc. should be sent to
% Adrienne Griscti (griscti@acm.org)
%
% Technical questions _only_ to
% Gerald Murray (murray@hq.acm.org)
% ===============================================================
%
% For tracking purposes - this is V2.0 - May 2012


\documentclass{sig-alternate}

\usepackage{verbatim}
\usepackage{algorithm}
\usepackage{algorithmic}
\newcommand{\codefont}[1]{\footnotesize{\texttt{#1}}\normalsize}

%\usepackage{listings}
%\usepackage{color}
%
%\definecolor{dkgreen}{rgb}{0,0.6,0}
%\definecolor{gray}{rgb}{0.5,0.5,0.5}
%\definecolor{mauve}{rgb}{0.58,0,0.82}
%
%\lstset{frame=tb,
%  language=Java,
%  aboveskip=3mm,
%  belowskip=3mm,
%  showstringspaces=false,
%  columns=flexible,
%  basicstyle={\small\ttfamily},
%  numbers=none,
%  numberstyle=\tiny\color{gray},
%  keywordstyle=\color{blue},
%  commentstyle=\color{dkgreen},
%  stringstyle=\color{mauve},
%  breaklines=true,
%  breakatwhitespace=true
%  tabsize=3
%}


\begin{document}
%
% --- Author Metadata here ---
%\conferenceinfo{WOODSTOCK}{'97 El Paso, Texas USA}
\CopyrightYear{2014} % Allows default copyright year (20XX) to be over-ridden - IF NEED BE.
%\crdata{0-12345-67-8/90/01}  % Allows default copyright data (0-89791-88-6/97/05) to be over-ridden - IF NEED BE.
% --- End of Author Metadata ---

\title{Clipboard: Speculative System Editing with Naming Recommendation}
%\subtitle{[Extended Abstract]
%\titlenote{A full version of this paper is available as
%\textit{Author's Guide to Preparing ACM SIG Proceedings Using
%\LaTeX$2_\epsilon$\ and BibTeX} at
%\texttt{www.acm.org/eaddress.htm}}}
%
% You need the command \numberofauthors to handle the 'placement
% and alignment' of the authors beneath the title.
%
% For aesthetic reasons, we recommend 'three authors at a time'
% i.e. three 'name/affiliation blocks' be placed beneath the title.
%
% NOTE: You are NOT restricted in how many 'rows' of
% "name/affiliations" may appear. We just ask that you restrict
% the number of 'columns' to three.
%
% Because of the available 'opening page real-estate'
% we ask you to refrain from putting more than six authors
% (two rows with three columns) beneath the article title.
% More than six makes the first-page appear very cluttered indeed.
%
% Use the \alignauthor commands to handle the names
% and affiliations for an 'aesthetic maximum' of six authors.
% Add names, affiliations, addresses for
% the seventh etc. author(s) as the argument for the
% \additionalauthors command.
% These 'additional authors' will be output/set for you
% without further effort on your part as the last section in
% the body of your article BEFORE References or any Appendices.

\numberofauthors{3} %  in this sample file, there are a *total*
% of EIGHT authors. SIX appear on the 'first-page' (for formatting
% reasons) and the remaining two appear in the \additionalauthors section.
%
\author{
% You can go ahead and credit any number of authors here,
% e.g. one 'row of three' or two rows (consisting of one row of three
% and a second row of one, two or three).
%
% The command \alignauthor (no curly braces needed) should
% precede each author name, affiliation/snail-mail address and
% e-mail address. Additionally, tag each line of
% affiliation/address with \affaddr, and tag the
% e-mail address with \email.
%
% 1st. author
%\alignauthor
%Ben Trovato\titlenote{Dr.~Trovato insisted his name be first.}\\
%       \affaddr{Institute for Clarity in Documentation}\\
%       \affaddr{1932 Wallamaloo Lane}\\
%       \affaddr{Wallamaloo, New Zealand}\\
%       \email{trovato@corporation.com}
%% 2nd. author
%\alignauthor
%G.K.M. Tobin\titlenote{The secretary disavows
%any knowledge of this author's actions.}\\
%       \affaddr{Institute for Clarity in Documentation}\\
%       \affaddr{P.O. Box 1212}\\
%       \affaddr{Dublin, Ohio 43017-6221}\\
%       \email{webmaster@marysville-ohio.com}
% 3rd. author
% Just remember to make sure that the TOTAL number of authors
% is the number that will appear on the first page PLUS the
% number that will appear in the \additionalauthors section.
}
\maketitle
\begin{abstract}

 Developers often make {\it systematic edits} that are similar but not identical edits to multiple locations for bug fixes, feature editions, refactoring and new API adaptation.  It is challenging to identify all relevant locations and perform consistent changes to different contexts without introducing new bugs.  
 
  To help recommend similar systematic edits and complete code transformation automatically, we implement a tool called Clipboard to  identify all matched edit locations and perform consistent edits to all code locations. Our approach simplifies the configuration and invocation process for the code transformation via drag-and-drop.  With Clipboard, developers define the template of systematic edit  by dragging the example code region  to a virtual Clipboard before the editing and taking the snapshot again with drag-and-drop action after the editing. Developers can demonstrate one or more examples to generate an {\it edit recipe} \textendash a reusable template of complex edit operations.  
  
  
  Given an edit recipe, our tool automatically identifies all matched edit locations and applies the transformation to all code locations. Developers can also invoke Clipboard on the fly and our tool will recommend the most suitable recipe based on the incoming edit stream  against multiple edit recipes.  
    Before applying the changes to target edit locations,  Clipboard analyzes the naming patterns between related variables and recommends a reasonable name for the identifiers in the target code location. Finally, our tool  proactively checks the compilation errors, informs developers of the consequences after performing the transformation with the change previews at multiple code locations, and applies the transformation after approved by the developers.  
 
 We evaluate our tool with a test suite of 68 exemplar changed methods from Eclipse JDT and SWT. 
 
 
 % Our tool is able to    
  
%  edit locations based on the edit recipe or identify , generate reasonable variable names based on the naming patterns between related variables, and proactively check the compilation errors before applying the transformation. 

%Integrated development environments provide recommendations for a number of common edits, such as refactorings and fixes for compilation errors. However, these tools are confined with pre-defined program transformation templates with complex configuration. As the number of supported systematic edits and advanced options increases, invoking the consequence of the transformations becomes even harder and thus impedes the usage of systematic edits recommendation.  

%,  we build a new content assist Eclipse plugin, called SyditRecommender, for context sensitive systematic change recommendation and auto-completion upon user request using edit recipes learned from examples. Our approach simplifies the configuration and invocation process for the code transformation. Given one or more examples for the systematic edits, our tool first dynamically abstracts edit recipes with context information, detects incoming edit streams, recommends the most suitable transformation proposals and displays the  change preview before developers apply the transformation. SyditRecommender also provides a speculative analysis for the target programs to evaluate the consequence of applying the edit recipes. 
%direct drag-and-drop manipulation of code snippets. 
 %With Clipboard, developers can define reusable template of arbitrary edit operations by taking snapshot before and after the change with drag and drop action, preview the consequence of the edits and further apply the change by directly invoking the content assist completion engine in Eclipse or simply drag the snapshot snippet to the target piece of code. 

%Our preliminary result shows a prototype Eclipse content assist Plugin for the program transformation with edit recipes learned from examples. 


\end{abstract}

% A category with the (minimum) three required fields
%\category{H.4}{Information Systems Applications}{Miscellaneous}
%A category including the fourth, optional field follows...
\category{D.2.3}{Software Engineering}{Coding Tools and Techniques}[Program editors]

\terms{Search-based software engineering, Experimentation}

\keywords{Program Transformation, Speculative Analysis, Systematic Edit, Naming Pattern }

\section{Introduction}


\section {Related Work}\label{sec:related}
%\noindent {\textbf{Example-based Systematic Editing} } 
%J.Andersen et al. \cite{andersen:generic, andersen:semantic}

%%start from related work
\noindent {\textbf{Naming Pattern}}
Lawrie et.al \cite{lawrie:studyindentifier06} conduct a  comprehensive study of identifier names  and they find that there is no statistical difference between full words and abbreviations in many cases. Caprile et.al. \cite{caprile:reconstruct00} reconstruct identifier names based on both  standard lexicon and standard syntax, yet it is confined to pre-defined naming standards. 

Butler et.al.  \cite{butler:classname11}  conduct an empirical study of  conventional  Java class naming patterns and patterns of class names related to inheritance by identifying common grammatical structures of Java class identifiers.  Singer et.al. \cite{singer:pattern08} consider semantic information encoded in Java class names  and create  a prototype to inform programmers of particular problems or optimization opportunities in their code based on the naming patterns.
H{\o}st et.al. \cite{einar:debugname09} exploit  whether or not a method name and implementation are likely to be good matches for each other and provide a simple pattern-based naming recommendation approach to evaluate reasonable names for the identifiers.  Kashiwabara et.al \cite{yuki:verbrecommend14} recommend candidate verbs for method names based on the association rules extracted from similar code fragments and help developers consistently use various verbs in method names. Aligned with the idea of extracting association rules from similar code fragments, we construct recommended variable names based on the related contexts of the systematic edits. Our approach is able to generate multiple reasonable names for different edit locations based on the naming patterns.
%The functions include standard algorithms studied in computer science courses as well as functions extracted from production code. The results show that full word iden- tifiers lead to the best comprehension; however, in many cases, there is no statistical difference between full words and abbreviations.

%%% parameter and name recommendation
%GraPacc extracts API usage patterns with relevant context and control/data dependency. Once users query for the code completion, GraPacc extracts context features, rank the best matched pattern and fill in the code. Yet it requires developers manually input API usage pattern to extend its capability.  However, Cookbook assigns equal weight to each nodes that matches transformation template in syntax level without considering the relevance and popularity of different templates. 


%\noindent {\textbf{Naming Recommendation}}
%1. recommend method name+structure context matching+ IR concept location
%2. recommend verb name
%3. mining identifier naming

\section{Motivating Example}\label{sec:motivation}
\begin{figure}[ht]
\centering
\includegraphics[width=0.5\textwidth]{fig/codeexample.png}
\caption{Motivation Example from org.eclipse.compare.CompareEditorInput between v20061120 and v20061218. 
%The code snippets shown in red background show similar changes made to two different methods getActionBars() and getServiceLocator(). 
}
\label{fig:motivateExample}
\end{figure}

This section overviews the workflow of our tool based on a systematic edit  from  \codefont{org.eclipse.compare.CompareEditor} \codefont{Input} between  \codefont{v20061120 and v20061218}.  Figure~\ref{fig:motivateExample} displays part of the diff patch of this change, the code in black remains unchanged while the added code is illustrated in blue with  \codefont{"+"} ahead of line and the deleted statements is in red with  \codefont{"-"}. The function  \codefont{getActionBars()} and  \codefont{getServiceLocator()} in the class  \codefont{CompareEditorInput} utilize very similar but not identical context with a similar edit as well: removing a condition statement, adding an initialization sentence and add another condition statement afterward. The only difference of this change is the type of the variables. Given this similar yet not identical changes, SyditRecommender first generates an abstract edit transformation recipe for further use. %The backend recommendation detector will start to locate all similar edits in package level and count the occurrence of the systematic edits. 

Later, when developers type in some edits in a similar context, our tool invokes context assist engine in Eclipse and displays the recommendation options for all suitable transformation rules sorted in relevance and popularity order. When developers go through the options in the pop-up menu, our tool provides transformation preview on the target program as well as the consequence of this code change, that is, which errors will be invoked once applying this change. After checking the change preview and corresponding consequence evaluation, developers simply click the recommendation options and perform the edits on the target program.  


\section{Approach}\label{sec:approach}



\section{Evaluation}\label{sec:evaluate}

\section{Threats to Validity}\label{sec:validity}

\noindent {\textbf{Internal Validity}} The experience difference in two group could have affected the response time and result of the evaluation. %Participants are encouraged to perform the program transformation proposals in the experimental group. 
%\textbf{Construct Validity} 


\noindent {\textbf{External Validity}}  All the participants are graduate or undergraduate students in University of Texas at Austin major in Computer Science or Software Engineering. Although the participants have diverse experiences in Java, refactoring and Eclipse IDE, they might not be representative of all software developers in industry. Our recommendation tool is also confined with the Eclipse and supports Java only. 

\section{Discussion and Future Work}\label{sec:discussion}


\section{Conclusion}\label{sec:conclusion}
In this paper, we present the preliminary result of our IDE recommendation for systematic edits using edit recipes learned from examples. We describe our approach to match, filter, evaluate the consequence of the program transformation  and display the candidate proposal list with content assist engine in Eclipse with preview and consequence preview aside. Our tool eliminates the tedious invoking and configuration process for the program transformation with a safe consequence evaluation of potential compilation errors for the systematic change. Our experiment on the Eclipse SWT project illustrates that SyditRecommender is able to detect and apply the similar but not identical systematic edits with an accuracy of X\%. The user study we conduct prove that SyditRecommender  is more intuitive than the traditional program transformation tools with a X\% efficiency improvement on average. 




%\end{document}  % This is where a 'short' article might terminate

%ACKNOWLEDGMENTS are optional
%\section{Acknowledgments}


%
% The following two commands are all you need in the
% initial runs of your .tex file to
% produce the bibliography for the citations in your paper.
\bibliographystyle{abbrv}
\bibliography{sigproc,transform,clone,proactive,synthesis}  % sigproc.bib is the name of the Bibliography in this case
% You must have a proper ".bib" file
%  and remember to run:
% latex bibtex latex latex
% to resolve all references
%
% ACM needs 'a single self-contained file'!
%
%APPENDICES are optional
%\balancecolumns
%\appendix
%%Appendix A
%\section{Headings in Appendices}
%The rules about hierarchical headings discussed above for
%the body of the article are different in the appendices.
%In the \textbf{appendix} environment, the command
%\textbf{section} is used to
%indicate the start of each Appendix, with alphabetic order
%designation (i.e. the first is A, the second B, etc.) and
%a title (if you include one).  So, if you need
%hierarchical structure
%\textit{within} an Appendix, start with \textbf{subsection} as the
%highest level. Here is an outline of the body of this
%document in Appendix-appropriate form:
%\subsection{Introduction}
%\subsection{The Body of the Paper}
%\subsubsection{Type Changes and  Special Characters}
%\subsubsection{Math Equations}
%\paragraph{Inline (In-text) Equations}
%\paragraph{Display Equations}
%\subsubsection{Citations}
%\subsubsection{Tables}
%\subsubsection{Figures}
%\subsubsection{Theorem-like Constructs}
%\subsubsection*{A Caveat for the \TeX\ Expert}
%\subsection{Conclusions}
%\subsection{Acknowledgments}
%\subsection{Additional Authors}
%This section is inserted by \LaTeX; you do not insert it.
%You just add the names and information in the
%\texttt{{\char'134}additionalauthors} command at the start
%of the document.
%\subsection{References}
%Generated by bibtex from your ~.bib file.  Run latex,
%then bibtex, then latex twice (to resolve references)
%to create the ~.bbl file.  Insert that ~.bbl file into
%the .tex source file and comment out
%the command \texttt{{\char'134}thebibliography}.
%% This next section command marks the start of
%% Appendix B, and does not continue the present hierarchy
%\section{More Help for the Hardy}
%The sig-alternate.cls file itself is chock-full of succinct
%and helpful comments.  If you consider yourself a moderately
%experienced to expert user of \LaTeX, you may find reading
%it useful but please remember not to change it.
%%\balancecolumns % GM June 2007
%% That's all folks!
\end{document}
