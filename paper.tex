% This is "sig-alternate.tex" V2.0 May 2012
% This file should be compiled with V2.5 of "sig-alternate.cls" May 2012
%
% This example file demonstrates the use of the 'sig-alternate.cls'
% V2.5 LaTeX2e document class file. It is for those submitting
% articles to ACM Conference Proceedings WHO DO NOT WISH TO
% STRICTLY ADHERE TO THE SIGS (PUBS-BOARD-ENDORSED) STYLE.
% The 'sig-alternate.cls' file will produce a similar-looking,
% albeit, 'tighter' paper resulting in, invariably, fewer pages.
%
% ----------------------------------------------------------------------------------------------------------------
% This .tex file (and associated .cls V2.5) produces:
%       1) The Permission Statement
%       2) The Conference (location) Info information
%       3) The Copyright Line with ACM data
%       4) NO page numbers
%
% as against the acm_proc_article-sp.cls file which
% DOES NOT produce 1) thru' 3) above.
%
% Using 'sig-alternate.cls' you have control, however, from within
% the source .tex file, over both the CopyrightYear
% (defaulted to 200X) and the ACM Copyright Data
% (defaulted to X-XXXXX-XX-X/XX/XX).
% e.g.
% \CopyrightYear{2007} will cause 2007 to appear in the copyright line.
% \crdata{0-12345-67-8/90/12} will cause 0-12345-67-8/90/12 to appear in the copyright line.
%
% ---------------------------------------------------------------------------------------------------------------
% This .tex source is an example which *does* use
% the .bib file (from which the .bbl file % is produced).
% REMEMBER HOWEVER: After having produced the .bbl file,
% and prior to final submission, you *NEED* to 'insert'
% your .bbl file into your source .tex file so as to provide
% ONE 'self-contained' source file.
%
% ================= IF YOU HAVE QUESTIONS =======================
% Questions regarding the SIGS styles, SIGS policies and
% procedures, Conferences etc. should be sent to
% Adrienne Griscti (griscti@acm.org)
%
% Technical questions _only_ to
% Gerald Murray (murray@hq.acm.org)
% ===============================================================
%
% For tracking purposes - this is V2.0 - May 2012


\documentclass{sig-alternate}

\usepackage{verbatim}
\usepackage{algorithm}
\usepackage{algorithmic}
\newcommand{\codefont}[1]{\footnotesize{\texttt{#1}}\normalsize}

%\usepackage{listings}
%\usepackage{color}
%
%\definecolor{dkgreen}{rgb}{0,0.6,0}
%\definecolor{gray}{rgb}{0.5,0.5,0.5}
%\definecolor{mauve}{rgb}{0.58,0,0.82}
%
%\lstset{frame=tb,
%  language=Java,
%  aboveskip=3mm,
%  belowskip=3mm,
%  showstringspaces=false,
%  columns=flexible,
%  basicstyle={\small\ttfamily},
%  numbers=none,
%  numberstyle=\tiny\color{gray},
%  keywordstyle=\color{blue},
%  commentstyle=\color{dkgreen},
%  stringstyle=\color{mauve},
%  breaklines=true,
%  breakatwhitespace=true
%  tabsize=3
%}


\begin{document}
%
% --- Author Metadata here ---
%\conferenceinfo{WOODSTOCK}{'97 El Paso, Texas USA}
\CopyrightYear{2014} % Allows default copyright year (20XX) to be over-ridden - IF NEED BE.
%\crdata{0-12345-67-8/90/01}  % Allows default copyright data (0-89791-88-6/97/05) to be over-ridden - IF NEED BE.
% --- End of Author Metadata ---

\title{Clipboard: Speculative System Editing with Naming Recommendation}
%\subtitle{[Extended Abstract]
%\titlenote{A full version of this paper is available as
%\textit{Author's Guide to Preparing ACM SIG Proceedings Using
%\LaTeX$2_\epsilon$\ and BibTeX} at
%\texttt{www.acm.org/eaddress.htm}}}
%
% You need the command \numberofauthors to handle the 'placement
% and alignment' of the authors beneath the title.
%
% For aesthetic reasons, we recommend 'three authors at a time'
% i.e. three 'name/affiliation blocks' be placed beneath the title.
%
% NOTE: You are NOT restricted in how many 'rows' of
% "name/affiliations" may appear. We just ask that you restrict
% the number of 'columns' to three.
%
% Because of the available 'opening page real-estate'
% we ask you to refrain from putting more than six authors
% (two rows with three columns) beneath the article title.
% More than six makes the first-page appear very cluttered indeed.
%
% Use the \alignauthor commands to handle the names
% and affiliations for an 'aesthetic maximum' of six authors.
% Add names, affiliations, addresses for
% the seventh etc. author(s) as the argument for the
% \additionalauthors command.
% These 'additional authors' will be output/set for you
% without further effort on your part as the last section in
% the body of your article BEFORE References or any Appendices.

\numberofauthors{3} %  in this sample file, there are a *total*
% of EIGHT authors. SIX appear on the 'first-page' (for formatting
% reasons) and the remaining two appear in the \additionalauthors section.
%
\author{
% You can go ahead and credit any number of authors here,
% e.g. one 'row of three' or two rows (consisting of one row of three
% and a second row of one, two or three).
%
% The command \alignauthor (no curly braces needed) should
% precede each author name, affiliation/snail-mail address and
% e-mail address. Additionally, tag each line of
% affiliation/address with \affaddr, and tag the
% e-mail address with \email.
%
% 1st. author
%\alignauthor
%Ben Trovato\titlenote{Dr.~Trovato insisted his name be first.}\\
%       \affaddr{Institute for Clarity in Documentation}\\
%       \affaddr{1932 Wallamaloo Lane}\\
%       \affaddr{Wallamaloo, New Zealand}\\
%       \email{trovato@corporation.com}
%% 2nd. author
%\alignauthor
%G.K.M. Tobin\titlenote{The secretary disavows
%any knowledge of this author's actions.}\\
%       \affaddr{Institute for Clarity in Documentation}\\
%       \affaddr{P.O. Box 1212}\\
%       \affaddr{Dublin, Ohio 43017-6221}\\
%       \email{webmaster@marysville-ohio.com}
% 3rd. author
% Just remember to make sure that the TOTAL number of authors
% is the number that will appear on the first page PLUS the
% number that will appear in the \additionalauthors section.
}
\maketitle
\begin{abstract}

 Developers often make {\it systematic edits} that are similar but not identical edits to multiple locations for bug fixes, feature editions, refactoring and new API adaptation.  It is challenging to identify all relevant locations and perform consistent changes to different contexts without introducing new bugs.  
 
  To help recommend similar systematic edits and complete code transformation automatically, we implement a tool called Clipboard to  identify all matched edit locations and perform consistent edits to all code locations. Our approach simplifies the configuration and invocation process for the code transformation via drag-and-drop.  With Clipboard, developers define the template of systematic edit  by dragging the example code region  to a virtual Clipboard before the editing and taking the snapshot again with drag-and-drop action after the editing. Developers can demonstrate one or more examples to generate an {\it edit recipe} \textendash a reusable template of complex edit operations.  
  
  
  Given an edit recipe, our tool automatically identifies all matched edit locations and applies the transformation to all code locations. Developers can also invoke Clipboard on the fly and our tool will recommend the most suitable recipe based on the incoming edit stream  against multiple edit recipes.  
    Before applying the changes to target edit locations,  Clipboard analyzes the naming patterns between related variables and recommends a reasonable name for the identifiers in the target code location. Finally, our tool  proactively checks the compilation errors, informs developers of the consequences after performing the transformation with the change previews at multiple code locations, and applies the transformation after approved by the developers.  
 
 We evaluate our tool with a test suite of 68 exemplar changed methods from Eclipse JDT and SWT. 
 
 
 % Our tool is able to    
  
%  edit locations based on the edit recipe or identify , generate reasonable variable names based on the naming patterns between related variables, and proactively check the compilation errors before applying the transformation. 

%Integrated development environments provide recommendations for a number of common edits, such as refactorings and fixes for compilation errors. However, these tools are confined with pre-defined program transformation templates with complex configuration. As the number of supported systematic edits and advanced options increases, invoking the consequence of the transformations becomes even harder and thus impedes the usage of systematic edits recommendation.  

%,  we build a new content assist Eclipse plugin, called SyditRecommender, for context sensitive systematic change recommendation and auto-completion upon user request using edit recipes learned from examples. Our approach simplifies the configuration and invocation process for the code transformation. Given one or more examples for the systematic edits, our tool first dynamically abstracts edit recipes with context information, detects incoming edit streams, recommends the most suitable transformation proposals and displays the  change preview before developers apply the transformation. SyditRecommender also provides a speculative analysis for the target programs to evaluate the consequence of applying the edit recipes. 
%direct drag-and-drop manipulation of code snippets. 
 %With Clipboard, developers can define reusable template of arbitrary edit operations by taking snapshot before and after the change with drag and drop action, preview the consequence of the edits and further apply the change by directly invoking the content assist completion engine in Eclipse or simply drag the snapshot snippet to the target piece of code. 

%Our preliminary result shows a prototype Eclipse content assist Plugin for the program transformation with edit recipes learned from examples. 


\end{abstract}

% A category with the (minimum) three required fields
%\category{H.4}{Information Systems Applications}{Miscellaneous}
%A category including the fourth, optional field follows...
\category{D.2.3}{Software Engineering}{Coding Tools and Techniques}[Program editors]

\terms{Search-based software engineering, Experimentation}

\keywords{Program Transformation, Speculative Analysis, Systematic Edit, Naming Pattern }

\section{Introduction}

Recent studies show that developers repeat their own mistakes or unknowingly repeat the errors from other in similar but not identical contexts and they are leaning to reuse existed code fragments with adaptive changes for similar programming tasks to save development effort \cite{nguyen:evolve}.   Tools such as linked editing \cite{graham:linkedit04}, CloneTracker \cite{ekwa:clonetracker07} and Cleman \cite{nguyen:cleman08} are able to modify multiple code fragments as one yet they are confined to identical changes on clone regions only.   Other  API migration tools  \cite{nguyen:api, andersen:semantic, nguyen:graph}  focus on high-level stylized API transformation with adaptation patterns from codebase. 

%extract context features, rank the best matched pattern from the database and fill in the code automatically. 

Sydit \cite{meng:sydit} and LASE \cite{meng:lase} match codebase with the given systematic edit template,  abstract context-aware transformation from examples,  identify similar edit locations,  and apply adaptive changes to similar code segments.  However,   these tools are confined to matching the codebase against a single recipe and perform similar edits to all matching code locations. %cannot match a specific edit location with multiple edit recipes and identify the most suitable edit recipe.

 Cookbook \cite{john:cookbook}     actively matches the incoming edit stream with multiple edit recipes to recommend the most suitable transformation rule on the fly.  Yet  Cookbook cannot guarantee a safe transformation without introducing compilation errors because it evaluates the control and data dependence only when it generates the edit recipe but does not check whether the transformation will invoke new compilation errors to the target after transformation.  
 
 We proactively apply the transformation to the target code and invoke Eclipse compilation error checker to evaluate the consequence of the transformation before undoing the change and rolling back to the original version. Motivated by the speculative consequence evaluation \cite{muslu:offline}, we  inform developers of  the preview of the target context after transformation and the compilation errors caused by the transformation before the change is actually applied to the edit location. 
If the change will invoke any compilation errors, our tool asks for developer's confirm before applying the transformation. Developer can also select one or more edit locations in the recommendation list and review the preview of the transformation with corresponding compilation errors caused by the transformation. After reviewing the systematic edits in the target locations, the developer is able to apply the systematic edit to one or multiple locations, to all the places that will not invoke compilation errors, or to all places without considering compilation errors. 


Similar to Cookbook, our tool matches the suitable edit recipe by tracking living edit stream. Yet in Clipboard, when the developer invokes code completion engine in Eclipse and selects an edit recipe from the pop-up menu, our tool provides the  preview of the change with potential compilation errors at the area of additional information next to the pop-up recommendation menu.


To further simplify  the process of generating, selecting and invoking the program transformation rule,   Clipboard provides an intuitive drag-and-drop approach to demonstrate the snapshots of the example code region before and after the transformation, and applies the transformation by double click or dragging the item of the example (or edit recipe) to the target context. Once the developer drags the code region to the Clipboard, our tool matches similar context based on the similarity of the syntactic structure and recommend the locations that might require similar systematic edit. After the developer specifies the change by dragging the  changed code region and dropping it to the new version of the edit recipe in the Clipboard, our tool generates the transformation result of the top ranked edit location and displays the preview of the systematic edit to the developers. Developers can select one or more candidate edit locations and apply the systematic edits to multiple edit locations.   

%Our approach exceeds these systematic program transformation tools in the following three aspects. First, these tools match the codebase against a single recipe to perform similar edits to all matching code locations, while Clipboard is able to actively match the incoming edit stream with multiple edit templates to recommend the most suitable transformation rule. Second, our tool provides transformation preview to the developers, proactively checks the compilation error caused by the transformation, and warns developers before the transformation is performed. Finally, as Sydit and LASE abstract the identifiers in the edit scripts, they fail to give a meaningful name to the parameters in the target code region.  ClipBoard overcomes the naming problem with a transformation mapping dictionary generated from syntactic difference \cite{fluri:distiller07} between the given example(s) and the target region, and recommend reasonable names to the corresponding variables.

To leverage the comprehension of systematic edits, we record the values of the matched variables and methods between the example and the target location, extract association rules, and synthesize the recommended names for the variables in the target location. For instance, given a new inserted statement 
\codefont{IActionBar actionBar = fContainer.getActionBars()}
in the example, our approach extracts the differences from the match pair of variable types \codefont{(IActionBar, IServiceLocator)} and the match pair of method calls \codefont{(fContainer.getActionBars(), fContainer.getServiceLocator())}, and automatically recommends the  name of the new variable \codefont{\$v} as \codefont{serviceLocator} in the statement \codefont{IServiceLocator \$v = fContainer.getServiceLocator()}.   

%association rules from matched types and variable and identify
%, simplify the selection and invocation of program transformation rules, and inform developers of the potential compilation errors from the transformation, we construct an Eclipse Plugin to support systematic edits and recommend the most suitable edit recipes based on the context of the incoming edit streams. 



%developers defines particular edit recipes by demonstrating the snapshot before the change and the one after applying the transformation  



%before applying the edits to the program. With these proposals, our tool monitors the incoming edit stream against the contexts of existing recipes in syntax level. Meanwhile, our tool detects the occurrence of the similar systematic edits for each recipe stored in an edit script library and ranks the suitable recipes considering both the similarity and occurrence of the templates. Common edit operations in ranked recipes are used to further filter out irrelevant transformation rules. Then the top ranked scripts are displayed as content assist proposals in Eclipse Engine.  Once developers select a recipe from the pop-up menu of transformation recommendation, the tool automatically displays the change preview of the program. To help evaluate the consequence of program transformation, our tool keeps a separate copy for the file under editing at the background and checks the compilation errors with context dependency for this program transformation without disturbing the current workspace. This file mirror is continuously synchronized with the editing file after each editing keystroke once the content assist is invoked in Eclipse IDE. Developer simply clicks the edit script option in the pop-up menu and apply the transformation on the target program after checking the preview and the transformation consequence.
%SyditRecommender then applies the transformation rule on the copy file and analyzes the consequence of the code change before displaying the consequence along with the change preview.  



%We evaluate our tool with 68 exemplar changed method drawn from the version history of Eclipse SWT from prior work \cite{meng:lase, john:cookbook} .  

To evaluate the usability, intuitiveness and effectiveness for this code recommendation system, we conducted an initial user study with 68 exemplar changed method drawn from the version history of Eclipse SWT from prior work \cite{meng:lase, john:cookbook} .  On average, our tool spent Z seconds to generate edit script and perform the transformation to the edit location which does not invoke compilation error, compared to W seconds in manual edit effort without the tool support. Our consequence evaluation correctly addressed X out of Y systematic edits with an accuracy of X\%. Compared to the edit version history in the code repository, our approach correctly recommends X out of Y names of the new identifiers. Our result shows that Clipboard is able to speed up the program transformation and minimize the omission errors in program transformation. 

%The edits generated from the selected recipe are Z\% similar to the developers' manual changes. 
%We also found that our consequence evaluation has a precision of X\% and recall of Y\%.  
% after  locate all relevant edit locations, X milliseconds to find the suitable recipes on the fly,  and Y milliseconds to generate the  transformation code based on the template and target location, By comparing the automatic transformation result with the manual edits in the version history, 
%To evaluate whether recommendations of relevant edit locations are of any help to developers working on a change task, and if  there are recommendations developers find useful but that Clipboard does not suggest
 %on X developers from University of Texas at Austin with X years development experience on average in Object-oriented languages such as Java and C\#. 
% By separating them into experimental group who adopted Clipboard and control group without any tool support. We found that Clipboard is able to improve developers efficiency of Z\% on average with a better confidence to apply the transformation automatically. Z\% of them agreed that Clipboardwas helpful and could improve the working efficiency and eliminate human errors in systematic editing.  By comparing the result of two groups, we found that SyditRecommender is able to generate systematic edits X\% similar to the manual editing.

In summary, our paper makes the following contributions. 
\begin{itemize}
  \item  To simplify the  definition, invocation, and configuration process of example-based systematic edits, we allow developers to manipulate the region of systematic edits intuitively via drag-and-drop. % eliminate the need for menu or dialogs, and allow developers to directly manipulate  . 
Given a list of edit recipes, we match the most suitable systematic edit on the fly and invoke the transformation based on the incoming edit stream. 


 % By ranking the context in syntax level and considering the occurrence of the similar systematic edits, we build an intuitive content assist Eclipse Plugin to recommend the most suitable systematic recipes using edit scripts learned from examples. SyditRecommender simplifies the obscure configuration and invocation process of the code completion. 
  \item  We conduct a speculative analysis of the potential compilation errors caused by the transformation and inform the developer of the potential errors along with the preview of the target code region after the systematic edit.  Our approach proactively applies the transformation to the target location, evaluates the compilation error invoked by the change, and rollbacks to the original version. 
  
  \item To increase the comprehension of the auto-generated transformation code, we leverage systematic naming patterns in program differences and synthesize names of the corresponding identifiers in the target edit location.
 % We propose a speculative consequence evaluation on the potential recipe adaptation by applying the edits on a hidden copy of the active file, compiling it at the background and providing potential compilation errors and dependency conflicts to the front end. 
%  \item We provide a preview of the program transformation after applying the selected edit recipe to increase user's confidence on applying the systematic edits. 
%  \item In our user study, X\% participants agreed that SyditRecommender were helpful to the systematic edits and were able to improve programming efficiency in an intuitive way. Our experiment on a medium size project shows that our technique achieves an accuracy of A\% and B\% faster than manual edits adaptation with Z\% similarity to the manual changes.
\end{itemize}

The rest of the paper is organized as follows. Section~\ref{sec:related} places our tool in the context of related works in code recommendation and systematic edit transformation. Section~\ref{sec:motivation} overviews our approach with an motivation example. Section~\ref{sec:approach} describes the concrete steps for our tools with matching, filtering and consequence evaluation algorithms. Section~\ref{sec:result} presents the preliminary result for the UI prototype of the content assist Eclipse Plugin, followed by the preliminary evaluation in Section~\ref{sec:evaluate}. The next two sections discusses the threats to the validity, discussion and future work for our program transformation recommendation technique. And the paper concludes in Section~\ref{sec:conclusion}.

%Despite these code limited with pre-defined semantic preserving edits,  


%Other program transformation approaches \cite{bruch:rank, robbes:history} provide a better suggestion for the candidate list by means of relevance or version history. 


%%%move to introduction



%Recent work observes that a large amount of code changes are systematic, similar but not identical.   

%Program transformation tools always require tedious parameter prescription with a predefined template in advance. 

%Code completion helps improve developers' program productivity. When developers type in part of the code, the code completion engine is able to automatically fill the rest of code by matching editor's input stream with pre-defined templates. For example, the \textit{Completion Engine} in Eclipse provides \textit{call completion} for likely method calls whenever developers trigger code completion on an object  and \textit{quick fixer} for actions that can be undertaken to repair the error.  We  supports custom, reusable templates of complex edit operations. By specifying change examples, Cookbook first generates an abstract edit recipe to describe the most specific generation of the demonstrated example program transformations, before storing the recipe into the library. Then it matches a developer's edit stream and recommend a suitable program transformation script that is capable of filling out the rest of change customized to the target. However, it requires a custom edit recipe library to match developers' edit stream. If developers want to generate a new recipe, they have to train Cookbook with two or more examples before applying the recipe to the third or more edits. In this project, we propose to use version history in forked projects to train the change examples and apply similar changes from peer projects to the target one. We also plan to improve the user interface of the Cookbook and make it more user?friendly.

%Current program transformation tools are 



%, \textit{subwords completion} which matches part of the text instead of the exact prefixes 


% However, the current support for code completion is limited to pre-defined code templates or single method call of the variable to complete the rest of changes. There exists a code completion technique call Cookbook that supports custom, reusable templates of complex edit operations. By specifying change examples, Cookbook first generates an abstract edit recipe to describe the most specific generation of the demonstrated example program transformations, before storing the recipe into the library. Then it matches a developer's edit stream and recommend a suitable program transformation script that is capable of filling out the rest of change customized to the target. However, it requires a custom edit recipe library to match developers' edit stream. If developers want to generate a new recipe, they have to train Cookbook with two or more examples before applying the recipe to the third or more edits. In this project, we propose to use version history in forked projects to train the change examples and apply similar changes from peer projects to the target one. We also plan to improve the user interface of the Cookbook and make it more user?friendly.



%Integrated Development Environments help simplify common development tasks and thus improve the correctness and efficiency of software development. For instance, Eclipse facilitates auto-complete mechanism to recommend functions based on user's input stream, while Quick Fix provides possible recommendations to resolve compilation errors. Based on pre-defined templates, Omar et al. proposed an architecture to integrate highly-specialized code generation tools into developers' workflow while BeneFactor \cite{ge:benfactor} and WitchDoctor \cite{foster:witchdoctor} detect users' manual refactoring behavior on the fly and remind or complete the refactoring automatically with pre-defined code templates.  %Quick Fix Scout \cite{muslu:scout} evaluates the consequence of the  Quick Fix recommendations from Eclipse Engine.
%DNDRefactoring \cite{lee:dndrefactor} streamlines the invocation and configuration precess for refactoring by drag-and-drop program elements.  
%All of these code completion tools require pre-defined edit recipes to perform the transformations.  Nguyen et al. \cite{nguyen:graph} facilitate context-sensitive features to search and rank API usage patterns that are most fitted with the code under editing, yet developers need to provide textual template code fragments as usage pattern to extend code. 
%Sydit \cite{meng:sydit} and LASE \cite{meng:lase} match codebase and abstract context-aware transformation from examples and apply custom changes to similar code segments, yet it is not easy to select the most suitable edit recipe and developers are reluctant to use it without understanding the rationale and fully dependency of the code transformation. 

%there is a lack of consequence evaluation for the 

%developers have to specify textual examples with dialog and manually select the edits from recipe library. 

\section {Related Work}\label{sec:related}
%\noindent {\textbf{Example-based Systematic Editing} } 
%J.Andersen et al. \cite{andersen:generic, andersen:semantic}



%\textbf{Code Clone Detection}
%Prior research shows that developers often port similar features and bug fixes within and across projects~\cite{Nguyen:evolve, cordy:scatter, baishakhi:port}. 







\noindent{\textbf{Example-based Editing}}
%1. program transformation 
ClipBoard finds its roots in {\it programming by demonstration} (PBD), which is also called programming by examples \cite{lieberman:pbd}. In PBD, the user demonstrated one or more examples of a program, and the system generalizes the demonstration into a program that can be applied to other examples. Simultaneous editing \cite{miller:simultaneousedit01}  performs a group of repetitive text editing simultaneously with selection guessing \cite{miller:multiselect02}  and text constraints \cite{miller:lapis02}. Although the PBD text editors, such as Sublime \cite{sublime}, are able to provide multiple selection and batch editing functions to help developers, they can only  make the same textual changes to multiple locations simultaneously. Clipboard exploits program AST structure and performs similar but not identical systematic editing to multiple places. 


LIBSYNC \cite{nguyen:api} and spdiff \cite{andersen:generic, andersen:semantic}   detects the difference in API usage from multiple instance and migrate the programs by learning API usage adaptation patterns. Yet these API migration tools are confined to stylized API usage and fail to consider dependence constraints of the surrounding context. Our approach supports expressive and customizable transformation to multiple edit locations with customized context-awareness systematic edits to each location. 

Semdiff \cite{dagenais:semdiff} automatically recommends adaptive changes in the face of  high-level method additions and deletions, while our approach focuses on more fine-grained systematic transformation and recommend similar edits based on the syntactic structure similarity. Negara et.al \cite{johnson:minepattern14} mine fine-grained frequent code change patterns with the AST nodes but are unable to transform the similar change to multiple edit locations. PRECISE \cite{zhang:parameterrecommend12}  extracts usage pattern from existing API usage and adaptively recommend parameters based on the current context. 

Sydit  \cite{meng:sydit}  generates edit scripts from a single systematic edit example and requires developers to specify the target locations before applying the transformation. Whereas LASE  \cite{meng:lase}  overcomes the insufficient transformation rule generation from a single example and identifies the target transformation locations automatically. Different from these tools, Clipboard is able to actively match the incoming edit stream with multiple edit templates to recommend the most suitable transformation rule with compilation error checking. 



                
%since SYDIT and LASE extract edit scripts based on the edit operations (eg. insert and delete), they fail to handle incontinuous  context across functions. Our tool %Yet neither of these approaches provide consequence evaluation on the fly before adapting the transformation rules.


%2. Example based programming
%3. API migration
%4. Twinning
%5. Syndit, LASE

%\noindent{\textbf{ Edit location suggestion }}


%1. Suggest edit location
%2. Simultaneous Edit+ +LAPIS+multiple selection 


%5. Sublime


\noindent{\textbf{Code  clone management }}
%1. Clone Region Descriptor
%2.CCEvents
%2. Hot Clone
%3. CloneBoard
%linked editing
Recent research results pointed out that software system inevitably contain a large amount of similar code due to the copy-and-paste programming practice  \cite{cordy:scatter, su:clonebug07}. These similar or identical code fragments, called {\it code clones}, may impact software quality with inconsistent modifications made to the cloned code snippets \cite{gabel:inconsistence10}. CP-Miner \cite{li:cpminer06} uses data mining techniques to  detects copy-paste related bugs, Dejavu \cite{gabel:dejavu10} identifies syntactic inconsistency bugs based on the assumption that duplicated code fragments generally intend to remain identical, while  SPA \cite{baishakhi:spa13} facilitates state-control and data-dependence analysis to detect semantic inconsistency in ported code. Different from these clone-related bugs detection tools, ClipBoard focuses on performing semantic consistency transformation with compilation error checking rather than identifying inconsistency changes after the edits are applied to the codebase. 

Some clone management tools, such as linked editing \cite{graham:linkedit04}, Cleman \cite{nguyen:cleman08} and HotClones \cite{niko:hotclone12}  detect code clones, allow developers to modify multiple code clones as one, and track changes in separate clone groups. Yet none of these tools are able to apply context-awareness edit transformation.  Other clone management tools support live clone change tracking. CCEvents \cite{zhang:ccevent13} continuously monitors code repositories and provides timely notifications to the stakeholders based on contextual clone events.  CloneBoard \cite{wit:cloneboard09}  monitors copy-and paste activities in the Eclipse and offers resolution strategies for inconsistently modified clones, yet it is confined to the clipboard activity in the Eclipse IDE and cannot perform semantic consistence transformations within clone group. CloneTracker \cite{ekwa:clonetracker07} applies clone region descriptors to modify multiple sections of code consistently. Yet it fails to perform similar but not identical systematic edits based on the context dependency analysis. Different from these tools for clone-based change management, ClipBoard focuses on the syntactic similar but not identical code regions and performs adaptive changes to multiple edit locations. 

\noindent {\textbf{Program synthesis and code completion}}
%1. DND Refactoring -refactoring + Cookbook
%2. IDE track change
It is common that the programmer knows what type of object he
needs, but does not know how to write the code to get the object. Program synthesis tools, such as Jungloid Mining \cite{bodik:jungloid05} and CodeHint \cite{bodik:codehit14}, search for a chain of method calls to complete the calling path at runtime based on specifications. SemFix \cite{nguyen:semfix13} automatically repairs methods based on symbolic execution, constraint solving and program synthesis while MintHint  \cite{orso:minthint13}  identifies expression based on the repaired pattern and synthesize repair hints from these expressions. GenProg \cite{claire:genprog12}  is another automatic program repair tool which uses generic programming to synthesize potential bug-fixes based on test suites. Different from these search based program synthesis tools, our approach focuses on transforming systematic edits with syntactic consistency across multiple edit locations. 

%Other automatic program repair tools such as GenProg 

GraPacc \cite{nguyen:graph} extracts context features, rank the best matched pattern from the database and fill in the code automatically. Yet these tools are confined to the pre-defined specifications or API usage patterns while in our approach, developers can customize the systematic edit template on the fly before applying the transformation to the target code region. Our prior work, Cookbook \cite{john:cookbook}, automatically matches the incoming edit stream against multiple edit recipes, trying to find the most suitable recipe without requiring users to  specify the recipe they prefer to apply. Yet it does not recognize naming patterns between related types and variables, and fails to evaluation the consequence of the transformation based on the control dependency in the target code fragment. ClipBoard overcomes both these limitations of CookBook with a straightforward drag-and-drop script generation process.  
These interaction process is similar to the Drag-and-drop Refactoring \cite{lee:dndrefactor}, which improves the invocation and completion approach of refactoring in an intuitive manner but is confined to a set of pre-defined refactoring templates.  Some code completion tools such as WitchDoctor \cite{foster:witchdoctor} and BeneFactor \cite{ge:benfactor},  detect manual refactoring on the fly and automatically complete the code according to the command  from  the developers. 

Eclipse Content Assist \cite{eclipse:recommend} provides couple of auto-completion engines and suggests available methods based on live editing stream. Yet pre-defined transformation templates are required and Eclipse Content Assist does not consider the control and data dependency and thus might inject some errors when performing systematic edit transformation.  %Omar et.al. \cite{omar:specialize} integrate specialized code completion interfaces into a developer�s work�ow while 
%Bruch et.al. \cite{bruch:rank} and Robbes et.al. \cite{ robbes:history} improve the order of the code completion recommendation based on the code repository history. 



\noindent {\textbf{Speculative Transformation}}
Codebase Replication \cite{muslu:offline, muslu:implicate} maintains a copy of developer's codebase, makes that copy codebase available for offline analyses to run without disturbed by the developer, and  implicates the potential offline code changes  continuously. Another proactive analysis tool\textendash Quick Fix Scout \cite{muslu:scout}, aims to check the consequence of the quick fix recommendation in Eclipse platform and inform the developers of the potential conflicts after applying the quick fix. It maintains a hidden copy of the source file that is under editing and fore-apply the quick fix separately at the background without affecting the workspace.  ClipBoard focuses on systematic edit checking before it is actually applied to the source code, rather than general code changes or quick fix recommendations. Our approach can also proactively check multiple edit locations of the given transformation template.  

Different from change impact analysis tools such as Chianti \cite{ren:chianti05} that analyzes the changes between two versions of the applications based on the call graph for regression test selection, we consider control and data dependency when we generate the transformation code not when we check the consequence of the transformation. 
Our approach proactively applies the transformation to the source code and invokes Eclipse compilation error checker to identify compilation errors before undoing the change to the original version of the codebase. 

Our tool is also different from the harmfulness evaluation of the code clone \cite{wang:cloneevaluate12}, which uses  machine learning technique to investigate common characters that indicate potential harmfulness. We recommend developers with compilation errors that will be invoked after applying the transformation in a concrete manner rather than inform them of the high-level harmfulness analysis. 
% implicates the potential offline code changes  continuously and informs the developer of the potential changes as quickly as possible after the change is made. %Codebase Replication copies the developer's codebase, incrementally keeps this copy codebase in sync with the developer's codebase, makes that copy codebase available for offline analyses to run without disturbing the developer and without the developer's changes disturbing the analyses, and makes analysis results available to be presented to the developer.

  

%Aligned with these works on consequence evaluation, we conduct a more fine-grained impact evaluation on the program transformation results in our program transformation and code completion tool.
%3. impact analysis for changes
%3. Auto program repair 
%\noindent {\textbf{Programming Synthesis}} 
%1. program by demonstration , search based software engineering

%2. Semfix+Jungloid+CHESS+CodeHint+MintHint
%3. Portfolio: recommend relevant function
%4. parameter recommendation




%%start from related work
\noindent {\textbf{Naming Pattern}}
Lawrie et.al \cite{lawrie:studyindentifier06} conduct a  comprehensive study of identifier names  and they find that there is no statistical difference between full words and abbreviations in many cases. Caprile et.al. \cite{caprile:reconstruct00} reconstruct identifier names based on both  standard lexicon and standard syntax, yet it is confined to pre-defined naming standards. 

Butler et.al.  \cite{butler:classname11}  conduct an empirical study of  conventional  Java class naming patterns and patterns of class names related to inheritance by identifying common grammatical structures of Java class identifiers.  Singer et.al. \cite{singer:pattern08} consider semantic information encoded in Java class names  and create  a prototype to inform programmers of particular problems or optimization opportunities in their code based on the naming patterns.
H{\o}st et.al. \cite{einar:debugname09} exploit  whether or not a method name and implementation are likely to be good matches for each other and provide a simple pattern-based naming recommendation approach to evaluate reasonable names for the identifiers.  Kashiwabara et.al \cite{yuki:verbrecommend14} recommend candidate verbs for method names based on the association rules extracted from similar code fragments and help developers consistently use various verbs in method names. Aligned with the idea of extracting association rules from similar code fragments, we construct recommended variable names based on the related contexts of the systematic edits. Our approach is able to generate multiple reasonable names for different edit locations based on the naming patterns.
%The functions include standard algorithms studied in computer science courses as well as functions extracted from production code. The results show that full word iden- tifiers lead to the best comprehension; however, in many cases, there is no statistical difference between full words and abbreviations.

%%% parameter and name recommendation
%GraPacc extracts API usage patterns with relevant context and control/data dependency. Once users query for the code completion, GraPacc extracts context features, rank the best matched pattern and fill in the code. Yet it requires developers manually input API usage pattern to extend its capability.  However, Cookbook assigns equal weight to each nodes that matches transformation template in syntax level without considering the relevance and popularity of different templates. 


%\noindent {\textbf{Naming Recommendation}}
%1. recommend method name+structure context matching+ IR concept location
%2. recommend verb name
%3. mining identifier naming

\section{Motivating Example}\label{sec:motivation}
\begin{figure}[ht]
\centering
\includegraphics[width=0.5\textwidth]{fig/codeexample.png}
\caption{Motivation Example from org.eclipse.compare.CompareEditorInput between v20061120 and v20061218. 
%The code snippets shown in red background show similar changes made to two different methods getActionBars() and getServiceLocator(). 
}
\label{fig:motivateExample}
\end{figure}

This section overviews the workflow of our tool based on a systematic edit  from  \codefont{org.eclipse.compare.CompareEditor} \codefont{Input} between  \codefont{v20061120 and v20061218}.  Figure~\ref{fig:motivateExample} displays part of the diff patch of this change, the code in black remains unchanged while the added code is illustrated in blue with  \codefont{"+"} ahead of line and the deleted statements is in red with  \codefont{"-"}. The function  \codefont{getActionBars()} and  \codefont{getServiceLocator()} in the class  \codefont{CompareEditorInput} utilize very similar but not identical context with a similar edit as well: removing a condition statement, adding an initialization sentence and add another condition statement afterward. The only difference of this change is the type of the variables. Given this similar yet not identical changes, SyditRecommender first generates an abstract edit transformation recipe for further use. %The backend recommendation detector will start to locate all similar edits in package level and count the occurrence of the systematic edits. 

Later, when developers type in some edits in a similar context, our tool invokes context assist engine in Eclipse and displays the recommendation options for all suitable transformation rules sorted in relevance and popularity order. When developers go through the options in the pop-up menu, our tool provides transformation preview on the target program as well as the consequence of this code change, that is, which errors will be invoked once applying this change. After checking the change preview and corresponding consequence evaluation, developers simply click the recommendation options and perform the edits on the target program.  


\section{Approach}\label{sec:approach}

\begin{figure}[ht]
\centering
\includegraphics[width=0.5\textwidth]{fig/sequence.png}
\caption{Workflow of SyditRecommender. The boundary of SyditRecommender is illustrated with dotted box}
\label{fig:sequence}
\end{figure}


Figure~\ref{fig:sequence} shows the workflow of SyditRecommender. Users can define new edit recipes by demonstrating examples to Recommendation Generator. Our tool applies an AST differencing algorithm to create an edit script and generates the most specific edit scripts. At the same time, our tool automatically investigates all similar edits in package level and stores both the edit recipe as well as the occurrence of the similar systematic changes to the Edit Script Library in the format of XML file. We extend the generation and detection mechanisms of edit scripts in Sydit \cite{meng:sydit} and LASE \cite{meng:lase} so that our tool can locate all similar edits in package level automatically once a new recipe is demonstrated.  The information we store in edit script library includes context information, edit operations and corresponding occurrence for each recipe. 

SyditRecommender monitors user's editing stream and simultaneously matches the context with all the recipes in the library. Section~\ref{sec:matcher} details the matching algorithm while Recommendation Filter in section~\ref{sec:filter} depicts how our tool filters out irrelevant recipes with respect to un-matching edit operations for different the scripts. Section~\ref{sec:checker} describes the mechanism we use to compute the consequence of the systematic transformation. 

Finally, we call the content assist engine (also known as code completion engine) in Eclipse to display the ranked candidate recipe list in a pop-up menu regarding the relevance and popularity. And developers can review the potential consequence and the corresponding preview of the selected program transformation rule. After comparing different edit recipes, the developers select the preferred one with a simple click on the menu option and the transformation rule is directly applied to the target program. 
More details of our tool are described below with respect to how SyditRecommender matches with the context, filters out the irrelevant recipes regarding the editing operations and evaluates the dependency conflicts and compilation errors that might cause after performing the preferred proposal.


\subsection {Recommendation Matcher}\label{sec:matcher}


\begin{algorithm}
\caption{Pseudocode For the Matcher}
\label{alg:match}
\begin{algorithmic} 
\STATE  Function \{getCandidateScripts\} \{context, script list\}
%$n \geq 0 \vee x \neq 0$
%\ENSURE $y = x^n$
 \FORALL {script $\in$  scriptLibrary} 
 \STATE weight := 0
  \STATE weight += forward search the concrete identifiers in the script 
    \STATE weight +=  forward search the generic identifiers in the script 
      \STATE weight +=  backward search the concrete identifiers in the script
  \STATE weight +=  backward search the generic identifiers in the script
   \STATE weight += occurrenceWeight of the script
  \IF {weight $> $ threshold}
  \STATE add script to the candidate list
  \ENDIF

    \ENDFOR
  \STATE sort the candidate list based on the matching weight
\RETURN candidate list
\STATE EndFunction

%\STATE  Function \{matchContext\} \{mTreeRoot, sTreeRoot, match\}
%%$n \geq 0 \vee x \neq 0$
%%\ENSURE $y = x^n$
%
% \FORALL {sNode: sTreeRoot.children} 
% \IF {sNode.hasChildren == NULL \AND match}
% \RETURN match
%\ELSE 
%\STATE    treeNode $\leftarrow$ mTree.breathFirstEnumeration()   
%\WHILE {treeNode.hasElements() }
%  \IF {node.matches(treeNode.nextElement) }
%  \RETURN matchContext(treeNode, sNode, TRUE)
%  \ENDIF
%   \ENDWHILE
%    \ENDIF
%\ENDFOR
\end{algorithmic}
\end{algorithm}

Given the input stream edits, SyditRecommender searches for the method context in each edit recipe in the script library. The goal of the matching algorithm shown in Algorithm~\ref{alg:match} is to find the most suitable scripts that matches for the current context in the sense that similar changes are more likely to happen between the code snippets with similar contexts. Aligned with matching technique in \cite{meng:sydit, meng:lase}, we conduct search for the concrete type, method and variable names exactly as it is, as well as the abstract identifiers by its syntax nodes. Here the concrete identifiers indicate concrete class name or keywords. For instance, in the statement  \codefont{Iterator it = new Iterator()}, the node \codefont{=},  \codefont{new}  and  \codefont{Iterator} are identical in all examples yet the variable name  \codefont{it} is different and is abstracted to a generic identifier  \codefont{ v\$0} .   By overriding the forward completion and backward completion functions in content assist engine, we analyze both upstream dependency and downstream dependency of the context. As a result, we get two lists of suggestions from edit templates. In the next step, we take the occurrence for each script into consideration and calculate the matching weight for each script. We finally rank all the candidate recipes whose matching weight is beyond the threshold. 


\subsection {Recommendation Filter}\label{sec:filter}
All the candidate scripts are further elaborated with the edit operations of \textit{\textbf{DELETE, INSERT, MOVE, UPDATE}} aligned with \cite{meng:sydit}. For example, the line 2 in Figure~\ref{fig:motivateExample} will be abstracted as  \codefont{ $e_{A} $ = \textbf{delete}(if (container == null )}. SyditRecommender keeps 2 versions of the editing file before and after each keystroke, and collects the editing operation for the edit stream, trying to match it with the candidate scripts. If the user is deleting the statement, then our tool will try to match the edit operations with the type of \textit{\textbf{DELETE}} in the candidate edit recipes. 
%The filter algorithm is precisely supported by the edit scripts 

\subsection {Recommendation Checker}\label{sec:checker}
We leverage the speculative analysis idea in Quick Fix Scount \cite{muslu:scout} to evaluate the potential consequence after applying the program transformation proposal. We keep a hidden version for the current editing file with synchronous update to the latest version. With this experimental copy, we safely apply the selected edit proposal and evaluate the consequence of this transformation before the transformation is applied to the real source file. We also facilitate the \textit{uodo change} in Eclipse API that rolls back the associated proposal application to recover the separate copy from one transformation rule and switch to the next one.    


\section{Preliminary Result}\label{sec:result}

%
%
%\begin{algorithm}
%\caption{Pseudocode For the Analyzer}
%\begin{algorithmic} 
%\STATE Function \{AnalyzeChange\} \{event, candidateScripts, oldDoc \}
% \STATE { sTreeRoot $\leftarrow$  XMLElement.getElement(script.ID)}
% \STATE {rangeDiff $\leftarrow$  Range.Differencer.getDiff(oldDoc,event.currentDoc)}
%\FORALL {script: candidateScripts}
%
%
%\IF {script.type ==TYPE.INSERT }
%\IF {rangeDiff.leftLength == 0 \AND rangeDiff.rightLength == 1}
%\STATE mathchLen$\leftarrow$sTreeRoot.getNodeValue("newNode"). matches (rangeDiff. getChangeText()) 
%\ELSE 
%\STATE continue
%\ENDIF
%
%\ELSIF {script.type ==TYPE.DELETE }
%\IF {rangeDiff.leftLength == 1 \AND rangeDiff.rightLength == 0}
%\STATE mathchLen$\leftarrow$sTreeRoot.getNodeValue("oldNode"). matches (rangeDiff. getChangeText())  
%\ELSE 
%\STATE continue
%\ENDIF
%
%\ELSIF {script.type == TYPE.MOVE }
%\STATE mathchLen$\leftarrow$sTreeRoot.getNodeValue("moveNode"). matches (rangeDiff. getChangeText())  
%
%
%\ELSIF {script.type == TYPE.UPDATE }
%\IF {rangeDiff.leftLength$\neq$0 \AND rangeDiff.rightLength $\neq$0}
%\STATE mathchLen$\leftarrow$sTreeRoot.getNodeValue("oldNode"). matches(rangeDiff.getChangeText())+sTreeRoot. getNode("newNode").matches(rangeDiff.getChangeText())  
%\ELSE 
%\STATE continue
%\ENDIF
%
% \IF {matchLen $> 0 $ }
% \STATE script.addWeight(matchLen*5)
%\ENDIF
%\ENDIF
%
%\ENDFOR
%\RETURN candidateScripts
%\STATE EndFunction
%\end{algorithmic}
%\end{algorithm}
%
%\begin{algorithm}
%\caption{Pseudocode For the ScriptApplyer}
%\begin{algorithmic} 
%\STATE Function {applyScript (script)}
%
%\IF {script.type ==TYPE.INSERT }
%\STATE   parentNode $\leftarrow$  script.getNode("parentNode")
%\STATE newNodesRoot $\leftarrow$ script.getNode("newNode")
%\STATE parentNode.insert(newNodesRoot);
%
%
%\ELSIF {script.type ==TYPE.DELETE }
%\STATE   rootNode $\leftarrow$  script.getNode("oldNode")
%\STATE   rootNode.removeFromParent()
%
%\ELSIF {script.type ==TYPE.MOVE }
%
% \STATE   rootNode $\leftarrow$  script.getNode("moveNode")
% \STATE  parentNode $\leftarrow$  script.getNode("newParent")
% \STATE parentNode.insert(rootNode)
%
%\ELSIF {script.type ==TYPE.UPDATE }
% \STATE   rootNode $\leftarrow$  script.getNode("oldNode")
% \STATE  newRootNode $\leftarrow$  script.getNode("newNode")
% \STATE  parentNode $\leftarrow$  script.getNode("parentNode")
%\STATE  rootNode.removeFromParent()
%\STATE parentNode.insert(newRootNode)
%\ENDIF
%\RETURN parentNode.toString()
%\STATE EndFunction
%\end{algorithmic}
%\end{algorithm}
%
%
%
%
%
%\begin{algorithm}
%\caption{Pseudocode For the ViewActionListener}
%\begin{algorithmic} 
%\STATE  {Function documentPreChange}
%\STATE {line $\leftarrow$  changedDocument.getChangedLine(event)}
%\IF{line $\neq$  lastChangedLine}
%\STATE {lastChangedLine $\leftarrow$ line}
%\ENDIF
%\STATE EndFunction
%
%\STATE {Function documentChange (event)}
%\STATE candidateScripts $\leftarrow$  Matcher.getCandidateScripts(event)
%\STATE   candidateScripts  $\leftarrow$   Analyzer.analyzeChange(event, candidateScripts, oldDoc)
%\STATE view.setInput(candidateScripts)
%\STATE EndFunction
%
%\STATE Function applyScript(script)
%\STATE  ScriptApplyer.applyScript(script)
%
%\STATE EndFunction
%
%
%\end{algorithmic}
%\end{algorithm}




\begin{figure}[ht]
\centering
\includegraphics[width=0.5\textwidth]{fig/contentassist.png}
\caption{Content Assist Example}
\label{fig:assist}
\end{figure}

We build a UI prototype for the Eclipse Content Assist Plugin. Figure~\ref{fig:assist} shows a snapshot of the SyditRecommender that recommends the potential edits proposals based on the editing context which has the features listed below:
 \begin{itemize}
 \item It ranks the candidate proposals with respect to its matching of the editing context and returns the sorted candidate recipe list.
 \item It caches the consequence of the transformation and warns users by displaying all compilation errors after applying the proposal on a hidden copy of the file under editing.
 \item It updates potential transformation proposals according to the editing stream and updates the consequence evaluation simultaneously.
 \end{itemize}

As shown in Figure~\ref{fig:assist}, SyditRecommender analyzes the editing context against all editing scripts in the library, computing the matching weight by looking forward and backward for both abstract identifiers as well concrete ones. It then ranks the candidate scripts with both matching weight and occurrence weight and invokes content assist engine in Eclipse to display the result in the pop-up menu shown in the graph. Simultaneously, SyditRecommender checks the consequence of each top ranked transformation rules and returns the compilation result for each recipe, so that whenever programer selects one of the proposals, our tool is able to show the preview of the transformation as well as the consequence evaluation aside in yellow comment area correspondingly. After checking the preview and consequence evaluation, developers selects to apply the preferred proposal with a click on the menu option.

\section{Evaluation}\label{sec:evaluate}
\subsection{Experiment on Eclipse SWT}

We first evaluate the precision and recall changes in terms of different matching threshold.  

We evaluate the accuracy and response time of SyditRecommender with 68 exemplar systematic edits from the version history of Eclipse SWT  aligned with the data set used in \cite{meng:lase}. We generate 28 edit recipes in the scripts library. The results are shown in Table X. We found that SyditRecommender correctly transforms X out of Y systematic edits correctly with a response time of X miniseconds on average. 

For the consequence evaluation, our approach correctly covers X compilation errors warning out of Z on average and Y control dependency and data dependency conflicts out of Z. 


\subsection{User Study}
To assess the usability, intuitiveness and efficiency of the SyditRecommender, we conduct a user study in X participants with Y years development experience in Object-oriented language such as Java and C\#. We created Z transformation tasks for the study and separate the participants in terms of its experience so that both groups have the same number of people with a similar working experience. One group is asked to use SyditRecommender while the other one preforms the systematic edits without any tool support. X\% of participants who use SyditRecommender agree that it is useful and can significantly improve the working efficiency with less human error in program transformation. We compared the time participants spent in two groups and found that the group with SyditRecommender is X seconds faster than the other one, with a X\% of improvement in efficiency. And the programs generated by SyditRecommender is Q\% similar to the corresponding human editing. 




\section{Threats to Validity}\label{sec:validity}

\section{Discussion and Future Work}\label{sec:discussion}


\section{Conclusion}\label{sec:conclusion}


%\end{document}  % This is where a 'short' article might terminate

%ACKNOWLEDGMENTS are optional
%\section{Acknowledgments}


%
% The following two commands are all you need in the
% initial runs of your .tex file to
% produce the bibliography for the citations in your paper.
\bibliographystyle{abbrv}
\bibliography{sigproc,transform,clone,proactive,synthesis}  % sigproc.bib is the name of the Bibliography in this case
% You must have a proper ".bib" file
%  and remember to run:
% latex bibtex latex latex
% to resolve all references
%
% ACM needs 'a single self-contained file'!
%
%APPENDICES are optional
%\balancecolumns
%\appendix
%%Appendix A
%\section{Headings in Appendices}
%The rules about hierarchical headings discussed above for
%the body of the article are different in the appendices.
%In the \textbf{appendix} environment, the command
%\textbf{section} is used to
%indicate the start of each Appendix, with alphabetic order
%designation (i.e. the first is A, the second B, etc.) and
%a title (if you include one).  So, if you need
%hierarchical structure
%\textit{within} an Appendix, start with \textbf{subsection} as the
%highest level. Here is an outline of the body of this
%document in Appendix-appropriate form:
%\subsection{Introduction}
%\subsection{The Body of the Paper}
%\subsubsection{Type Changes and  Special Characters}
%\subsubsection{Math Equations}
%\paragraph{Inline (In-text) Equations}
%\paragraph{Display Equations}
%\subsubsection{Citations}
%\subsubsection{Tables}
%\subsubsection{Figures}
%\subsubsection{Theorem-like Constructs}
%\subsubsection*{A Caveat for the \TeX\ Expert}
%\subsection{Conclusions}
%\subsection{Acknowledgments}
%\subsection{Additional Authors}
%This section is inserted by \LaTeX; you do not insert it.
%You just add the names and information in the
%\texttt{{\char'134}additionalauthors} command at the start
%of the document.
%\subsection{References}
%Generated by bibtex from your ~.bib file.  Run latex,
%then bibtex, then latex twice (to resolve references)
%to create the ~.bbl file.  Insert that ~.bbl file into
%the .tex source file and comment out
%the command \texttt{{\char'134}thebibliography}.
%% This next section command marks the start of
%% Appendix B, and does not continue the present hierarchy
%\section{More Help for the Hardy}
%The sig-alternate.cls file itself is chock-full of succinct
%and helpful comments.  If you consider yourself a moderately
%experienced to expert user of \LaTeX, you may find reading
%it useful but please remember not to change it.
%%\balancecolumns % GM June 2007
%% That's all folks!
\end{document}
